\documentclass{article}
\usepackage[margin=1.5in]{geometry}
\usepackage{amsmath}
\usepackage{indentfirst}

\title{6.176 Strategy Report}
\author{Str8Rippin \\
Erwin Hilton, Richard Spence, Gabriel Esayas}
\date{\today}

\begin{document}
\maketitle

\section*{Introduction}
This strategy report outlines the strategies that our pokerbot used for the 6.176 Pokerbots competition, including how we classified other opponents to determine whether to raise, check, call, or fold.

\section*{The \texttt{Opponent} class}
We created a Python class \texttt{Opponent} that keeps track of player stats throughout the game. Thus, whenever \texttt{NEWGAME} is received, we create three \texttt{Opponent} objects, including one for our own player. Each opponent has several basic attributes, including name, stack size, what seat the opponent is at (1, 2, or 3), whether the opponent is playing the current hand, and whether the opponent is eliminated from the game. Additionally, each \texttt{Opponent} has the following attributes: \\

\begin{tabular}{|l|l|}  \hline
\texttt{bot.foldPer} & percentage of hands folded pre-flop \\ \hline
\texttt{bot.folds} & number of folds pre-flop \\ \hline
\texttt{bot.VPIP} & percentage of hands in which opp. put money in pot pre-flop  \\ \hline
\texttt{bot.PFR} & percentage of hands in which opp. raised pre-flop \\ \hline
\texttt{bot.WTSD} & percentage of hands in which opp. went to showdown \\ \hline
\texttt{bot.WMSD} & percentage of showdowns in which opp. goes and wins \\ \hline
\texttt{bot.AFflop} & aggression factor - flop, defined as (\# bets+ \# raises)/(\# calls) \\ \hline

\end{tabular} \\

These attributes allow us to classify an opponent one of several categories: nit, shark, bomb, maniac, calling station, cashcow, loose, and unknown. 

%http://www.myholdempokertips.com/hud-stats-wtsd-won-money-at-showdown
%http://www.thepokerbank.com/articles/software/vpip/

We also use the M-ratio, so a player's M-ratio (denoted \texttt{bot.MRatio}) is defined by
\[ M = \frac{\text{stack}}{\text{small blind} + \text{big blind} + \text{total antes}} \]
However, the M-ratio simplifies in this game, since the antes are zero, and the small blind is always 1 and the big blind is always 2. Hence, the M-ratio is simply $\frac{\text{stack}}{3}$.

\section*{The equity calculator}
We used the given equity calculator, which takes in an input our hole cards, any community cards, as well as possibly the opponents' hole cards, and outputs a number in $[0,1]$ indicating the expected amount of the pot we should win with our hand.

Afterwards, we computed the estimated value ($EV$) for each hand, given by
\[ EV = \text{equity} \cdot \text{winAmt} - (1-\text{equity}) \cdot \text{loseAmt} \]
where winAmt equals the sum of the opponents' contributions to the pot, and loseAmt is the amount we have contributed to the pot.
\section*{Strategy}
At the end of each hand, we update the stats that were described above, for each player, if that person played that hand.

First, we parse the inputs from each \texttt{GETACTION} packet. Then we use the given equity calculator to determine the hand equity.

\subsection*{Preflop}
We mostly based our decision on our equity during preflop. If our equity is low, then we faced a few options:
\begin{itemize}
\item If a nit is raising, then he must have an excellent hand, so we fold.
\item Otherwise, with probability 0.35 and if the minimum bet amount is small, we bluff.
\item Otherwise, we check/fold.
\end{itemize}
Otherwise, if $EV > 0$, we look at equity. If our equity is larger than a threshold, we bet or raise. If $EV < 0$, we check/fold.

\subsection*{Postflop}
\end{document}